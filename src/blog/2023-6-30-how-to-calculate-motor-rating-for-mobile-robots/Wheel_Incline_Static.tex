% Author: Rsomething
% This document is created based on 'Izaak Neutelings' awesome examples of Tikz Friction/Torque
% I left the author's comments intact:
% Author: Izaak Neutelings (September 2020)
% Inspiration: https://tex.stackexchange.com/questions/25531/adding-underbrace-in-tikz
\documentclass[border=3pt,tikz]{standalone}
\usepackage{physics}
\usepackage{ifthen}
\usepackage{tikz}
\usepackage[outline]{contour} % glow around text
\usetikzlibrary{calc} % for pic
\usetikzlibrary{angles,quotes} % for pic
\usetikzlibrary{patterns}
\tikzset{>=latex} % for LaTeX arrow head
\contourlength{1.2pt}

\colorlet{myred}{red!65!black}
\tikzstyle{ground}=[preaction={fill,top color=black!10,bottom color=black!5,shading angle=20},
                    fill,pattern=north east lines,draw=none,minimum width=0.3,minimum height=0.6]
\tikzstyle{mass}=[line width=0.6,red!30!black,fill=red!40!black!10,rounded corners=1,
                  top color=red!40!black!20,bottom color=red!40!black!10,shading angle=20]

% FORCES SWITCH
\tikzstyle{force}=[->,myred,thick,line cap=round]
\tikzstyle{Fproj}=[force,myred!40]
\newcommand{\vbF}{\vb{F}}
\newboolean{showforces}
\setboolean{showforces}{true}

\begin{document}

% HORIZONTAL ground
\def\r{0.5} % mass radius

% INCLINED ground
\begin{tikzpicture}
  \def\W{3.5}  % ground width
  \def\D{0.2}  % ground depth
  \def\ang{30} % ground angle
  \def\mx{2.0} % mass x position
  \def\F{1.25} % force magnitude
  \draw[thick,top color=blue!20!black!30,bottom color=white,shading angle=\ang+10]
    (0,0) coordinate (O) -- (\ang:\W) coordinate (T) -- ({\W*cos(\ang)},0) coordinate (L) -- cycle;
  \draw[very thick,white,line cap=round] (0.5*\W,0) --++ (0.15*\W,0) (L)++(0,0.08*\W) --++ (0,0.06*\W);
  \draw[mass,rotate=\ang] (\mx,\r) circle (\r);
  \draw pic["$\theta$",draw=black,angle radius=22,angle eccentricity=1.3] {angle=L--O--T};
  
  % FORCES
  \ifthenelse{\boolean{showforces}}{
    \coordinate (F0) at ($(\ang:\mx-0.1*\r)+(\ang+90:0.2*\r)$);
    \coordinate (F)  at ($(F0)+(-90:\F)$);
    \coordinate (Fx) at ($(F0)+(\ang-180:{\F*sin(\ang)})$);
    \coordinate (Fy) at ($(F0)+(\ang-90:{\F*cos(\ang)})$);
    \draw[<->,rotate=\ang] (0.25*\W,0.38*\W) node[below=0,left=0] {$x$}
      -|++ (0.7*\r,0.7*\r) node[above left=-3] {$y$};
    \draw[force] (\ang:\mx)++(\ang+90:0.5*\r) --++ (\ang+90:{\F*cos(\ang)}) node[above] {$\vbF_\mathrm{N}$};
    \draw[force] (\ang:\mx+0.2*\r)++(\ang+90:0.2*\r) --++ (\ang:{\F*sin(\ang)}) node[above=-2] {$\vbF_\mathrm{f}$};
    \draw[dashed,myred!80!black!60] (Fx) -- (F) -- (Fy);
    \draw[Fproj] (F0) -- (Fx) node[left=-2]  {$mg\sin\theta$}; %\vu{x}
    \draw[Fproj] (F0) -- (Fy) node[right=-2] {$-mg\cos\theta$}; %\vu{y}
    \draw[force] (F0) -- (F)  node[below=-2] {$m\vb{g}$};
    \draw pic["$\theta$",draw=black,angle radius=12,angle eccentricity=1.5] {angle=F--F0--Fy};
  }{}
  
\end{tikzpicture}

\end{document}